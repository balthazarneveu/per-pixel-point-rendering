\documentclass[sigconf, nonacm]{acmart}
\settopmatter{printfolios=true}
\settopmatter{printacmref=false}
%% PACKAGES
\usepackage{graphicx}
\usepackage{hyperref}
\usepackage{cleveref}
\usepackage{subcaption}
\usepackage{natbib}
\usepackage{mathtools}
\usepackage{xcolor}

%% COLORS
\definecolor{darkgreen}{rgb}{0,0.8,0}

%% TITLES
\title{Review of ADOP: Approximate Differentiable One-Pixel Point Rendering}

%% AUTHORS
\author{Balthazar Neveu - ENS Paris-Saclay}
% \affiliation{%
%   \institution{ENS Paris-Saclay}
%   % \city{Saclay}
%   % \country{France}
% }
\email{balthazar.neveu@ens-paris-saclay.fr}


%% MAIN DOCUMENT
\begin{document}

  %% KEYWORDS
  \keywords{Neural rendering, Differentiable rendering, Point-based rendering}

  %% Teaser figure
  \begin{teaserfigure}
    \includegraphics[width=1.\textwidth]{figures/teaser_figure_2.png}
    \centering
    \caption{Overview of our partial re-implementation to study the ADOP \cite{ruckert2022adop} paper in ideal conditions with calibrated scenes. \\
      \textit{Left}: Point based neural rendering reconstructs novel view from a point cloud. Original paper implementation in ADOP  works on real photos of large scale scenes. It therefore tries to model camera exposure and non linear tone mapping to adapt to each camera rendering. \\
      \textit{Right}: We simulate real world photo captures by using calibrated scenes and cameras to be in the most simple conditions to review point based rendering. Point cloud is sampled randomly from the mesh with perfect normals, camera poses are perfectly known (these would usually be estimated using COLMAP \cite{schoenberger2016sfm}). We do not let exposure vary or perform tone mapping.
    }
    \label{fig:original_pipeline}
  \end{teaserfigure}

  %% TITLE
  \maketitle




  %% CONTENT
  \section{Introduction}
\label{sec:intro}
The purpose of this report is a review of the paper ADOP: Approximate Differentiable One-Pixel Point Rendering by \citet{ruckert2022adop}. 
Since the NERF paper published at ECCV 2020, there's been an incredible number of papers on neural rendering . Different approaches have been proposed with an underlying 3D data structure which allows rendering novel views of a scene. Neural radiance fields use a volumetric representation but other families of methods use a "proxy" such as a point clouds \cite{Aliev2020} or even meshes \cite{worchel2022nds}. \\
Let's put things simply: Point based rendering leads to images filled with holes and at first sight does not really look like an appropriate data structure to render continuous surfaces of objects.
We'll see how ADOP manages to use a point cloud structure jointly with an CNN (processing in the image space) to sample dense novel views of large real scenes.

A re-implementation from scratch in Pytorch of some of the key elements of the paper has been made in order to understand the most important points of the ADOP paper. To simplify the study, it seemed like a good idea to work on calibrated synthetic scenes. This way, we can focus on trying to evaluate the relevance of point based rendering and avoid the difficulties inherent to working with real world scenes, most nottably:
\begin{itemize}
    \item We assume linear RGB cameras without tone mappings.
    \item We discard environment map (e.g. our background is black).
    \item We generate photorealistic renders of synthetic meshes. 
    \item Camera poses are perfectly known.
    \item Using meshes allows us sampling point clouds with normals without estimation errors such as the one we'd face with COLMAP.
    \item We can easily control the number of points to be able to tests on limited capacity GPU.
\end{itemize}

\noindent Our code is available on ~\href{https://github.com/balthazarneveu/per-pixel-point-rendering}{GitHub}.

  \section{Context}
\label{sec:Context}

\textit{ECCV 2020 : The NERF \cite{mildenhall2020nerf} paper triggers an increasing interest in the field of novel views synthesis}. Can you render novel viewpoints from a given scene using a neural network instead of running a complex ray tracing engine (in a software such as Blender)? The answer is roughly yes and... it also works on real scenes where there's no knowledge of the underlying scene. Many works try to improve the quality/speed of the rendering. We'll see how, suprisingly, point clouds can be used to render novel views of a scene.


\subsection{What is novel view synthesis?}
\label{sec:novel_view_synthesis}


Novel view synthesis is a standard computer vision task which consists in generating new viewpoints of a scene after capturing a set of images.
When taking real photographs of a scene by walking around (or using a drone), the exact pose of the camera is not perfectly known. Even if an inertial measurement unit is attached to the camera which allows later to have an estimation of the camera pose, there will be measurement errors (sensor noise, calibration error, sensor fusion errors). \textit{Please not that efforts can be made, such as calibrating camera/IMU misalignment} \cite*{karpenko2011gyrostab}. 
So there's a need for an algorithm to estimate the camera poses from images, regardless of having an external pose estimation intialization.
The traditional pipeline consists in using Structure from Motion (SfM) like the popular COLMAP software \cite{schoenberger2016sfm} to jointly estimate camera trajectory. A side product of running this algorithm is getting a colored 3D point cloud of the scene. 
The second step is to reconstruct a flexible representation of the scene so it can be rendered from new viewpoint. The technical challenge is to find the most suited data structure to represent the scene subject to constraints such as: 
\begin{itemize}
    \item image / 3D structure quality: for cultural heritage applications for instance. 
    \item reconstruction time and memory consumption. Real time for AR/VR applications is a constraint. \textit{For instance, ADOP seems to take advantage of point cloud rendering hardware acceleration available in any computer using OpenGL (not necessarily with the need of a massive NVidia GPU)}. 
    \item preprocessing time: in case users want to recreate their own scenes, they may not have access to powerful GPU 
\end{itemize}
There's another usecase where the camera poses and scenes are perfectly known and controlled by using 3D scene synthesis. This is an easier setup to study novel view synthesis as you can truly evaluate the rendering quality of the algorithm without doubts on the quality of pose estimation (or camera photometry). This setup is sometimes refered as "calibrated scenes". One could say that representing scenes with sophisticated neural rendering is useless when you have the underlying 3D model and Blender available. Nevertheless, novel view synthesis on calibrated scenes is a good framework to test a method before deploying on real scenes.

\subsection{Representations of the scene}
\label{sec:representations}

If we solely use the colored point cloud, we'll end up with images filled with holes (when we zoom in for instance).
This is where rendering a scene start to get difficult. 
\noindent\textbf{Surface reconstruction.}It is possible to get continuous representations (without holes) by wrapping a surface around the point cloud of an object. For instance assuming we have pre-computed normals, a Signed Distance Function (SDF) defined from point cloud can be evaluated anywhere. The surface of the object is where the SDF is equal to zero. Evaluating IMLS (Implicit Moving Least Square \cite{kolluri2008IMLS}) on a fixed grid followed by the marching cube algorithm allows to recreate a mesh. All topologies may not be represented correctly (e.g. a hole in the cheese may end up being filled).


\begin{figure}[htbp]
    \centering
    \includegraphics[width=0.5\textwidth]{figures/material_appearance_commented.png}
    \caption{Colors change with camera orientation, specular materials reflect the light and amplify this effect, the extreme use-case being mirrors.}
    \label{fig:material_changes}
\end{figure}

\noindent\textbf{Meshes.} Creating a mesh from the point cloud will lead to a nice geometric representation which can be rendered using classic rasterization techniques. But shading these triangles is still needed. If all materials are perfect diffusers, applying the textures extracted from the photos to the triangles shall be enough. Unfortunately, this will not work for specular materials as illustrated in figure \ref{fig:material_changes}. Neural deferred shading has been proposed \cite{worchel2022nds} to jointly fit a mesh while optimizing a pixel shader (mimicked by a neural network) of a classic mesh rendering pipeline (geometry processing $\rightarrow$ rasterization $\rightarrow$ \textit{(neural)} shader). Please note that the shading is baked into the scene representation and cannot be changed afterwards (for instance, lighting or materials cannot be changed).
\noindent\textbf{Neural radiance fields.} NERF \cite{mildenhall2020nerf} represents the scene colors RGB and density as a function of the 3D position and viewing angle. By shooting rays at the scene, one can integrate the estimated colors to render the final color in the image (density=0 means that the space is empty). A MLP (Multi Layer Perceptron) can approximate basically any function as long as it has enough neurons... here the MLP is used to represent the radiance field. Having the dependance on the viewing angle allows to model complex materials and lighting effects. Main drawback is that NERF are computationally expensive as they require evaluating a MLP all along the rays for each pixel...
\noindent\textbf{Point clouds.} Although point clouds are not continuous, they are a good representation of the scene and kind of easy to manipulate. The introduction of an additional inpainting method (e.g. neural rendering component) in the image space allows to overcome the limitations of the sparseness. These will be discussed in more details the next section on neural point based graphics \ref{subsec:npbg}.



\section{Methodology of the Original Paper}
\label{sec:methodology_paper}

\subsection{Paper overview}
\label{subsec:paper_overview}
ADOP \cite{ruckert2022adop} is a novel view synthesis method based on point cloud representation. 

\noindent\textbf{Geometry.} Pseudo-colors  \footnote{Pseudo-colors means a generic "feature" vector representation which can be more generic than three dimensional RGB components. A dimension 4 for instance.} assigned to each point are projected onto the camera screen at several scales. An environment map is used to model the scene background.

\noindent\textbf{Neural rendering.} The multi-scale pyramid of pseudo images will be jointly decoded and inpainted into a RGB HDR \footnote{We define HDR as high dynamic range linear RGB images - as if they'd been retrieve from a RAW 12 or 14bit sensor with sole operations black point correction, demosaicing and potentially while balance/color matrix transforms.} image at full resolution, using a U-Net architecture.

\noindent\textbf{Camera simulation.} A camera simulation module will transform the image into a LDR image \footnote{LDR stands for low dynamic range images, the ones we see on our screens after tone mapping, color adjustments like vibrancy, vigetting correction etc...}. 

% TODO: ENVIRONMENT MAP!!!!!!!!!!!!!!!!!!!!!!!!!!!!!!!!!!!!!!!!!!!!!
\noindent The whole rendering pipeline is differentiable with regard to:
\begin{itemize}
    \item the pseudo-color of each point
    \item the environment map colors.
    \item the photometry camera parameters (exposure, white balance correction, tone curve parametric vignetting...)
    \item the camera pose and intrinsics \textit{This is an approximation, but it may be useful in order to refine camera pose estimation}. 
\end{itemize}

Using several photos of the scene for supervision, all parameters (network weights, pseudo-colors, environment map and camera photo pipeline) are optimized to minimize a loss function (MSE $\mathcal{L}^{2}$ or perceptual loss\footnote{Perceptual loss \cite{johnson2016perceptual} optimizes the distance between two images in a latent space rather than in the RGB colors space. We usually minimize the $\mathcal{L}^{2}$ distance between the feature maps in the middle of a frozen VGG network}) between predicted LDR rendered images and the real ones. 



\subsection{Neural Point Based Graphics}
\label{subsec:npbg}
NPBG \cite{Aliev2020} (Neural Point-Based Graphics) introduced several important concepts which will be discussed in the next section.
The idea of filling the holes between projected points by using multi-scale convolutional neural networks.  

\subsection{Point based Rendering}
\label{subsec:Point based Rendering}
One of the major advantage of point based methods is that this technology is backed by years of prior industy work. Rendering point clouds can be hardware accelerated and is available to the mainstream public on basically any computer (without NVidia GPU). Libraries such as OpenGL offer the option to render points \texttt{GL\_Points} (and even specify the size of the splat \texttt{glPointSize(1);})... Point clouds can even be rendered in a web browser as shown in figure \ref{fig:potree}.

% https://github.com/rougier/python-opengl/blob/master/07-points.rst
% https://glumpy.github.io/

\begin{figure}[htbp]
    \centering
    \includegraphics[width=0.5\textwidth]{figures/potree_rendering_and_splat.png}
    \caption{Potree \cite{potree} allows point cloud rendering in the browser, points being rendered as tiny circles splatted on the screen.}
    \label{fig:potree}
\end{figure}



\subsection{Novel view}
\label{subsec:Projecting points}



\noindent\textbf{Implementation.} The authors used lib Torch to compile the code. 


  \section{Reimplementation Strategy}
\label{sec:remplementation}

The main purpose of my implementation work was to be able to code by myself some of the core ideas of the ADOP differentiable point based neural renderer.
I decided to code everything from scratch knowing since the start that I would at no point be able to fully reproduce the results from the authors but I learnt a large amount of things by doing so.
\begin{itemize}
    \item I used BlenderProc \cite{Denninger2023} to script and generate multiple synthetic scenes from the samples used in the NerF paper. All camera positions are known and share the same referential as my pytorch point renderer.
    \item A perfect point cloud is sampled at random from the mesh (through the .obj file).
    \item A pytorch function allows to project the points onto the image plane and includes soft depth test and normal culling.
\end{itemize}

Each element has been carefully tested by mostly visual tests.






\subsection{Generating synthetic calibrated scenes}
\label{sec:synthetic_calibrated_scenes}
In order to avoid dealing with large real scenes with millions of points and having to deal with heavy COLMAP processing, I decided to use a recent library named BlenderProc  \cite{Denninger2023} which simplifies the use of Blender as it allows running renders in the background (launched through terminal without the need of a graphical user interface).

It is first possible to generate a set of viewpoints configurations to orbit around the scene. It is also possible to generate an environment map (e.g. a skybox). We render with the background transparency option so the rendering results is a RGB PNG. 
% 1532.9s for 60 views of the \texttt{material balls}.

Rendering is peformed using GPU and requires 25 seconds per view with a resolution 640x480 for the \texttt{material balls}. scene on a laptop Nvidia T500.

\begin{figure}[H]
    \centering
    \includegraphics[width=0.5\textwidth]{figures/blenderproc_renders.png}
    \caption{Mask and RGB render BlenderProc renders with an environment map.}
    \label{fig:blenderproc_renders}
\end{figure}


% 27 minutes are needed to generate 64 viewpoints
The environment map does not appear as a backround (although it is possible in the current code) as it would allow to simplify and remove one of the extra components of the ADOP paper (e.g. it allows rendering the point cloud without using the neural environment map trick).

We parameterize camera orientation using 3 Euler Angles (yaw pitch roll) and 3 positions. When we project a point onto the image plane, we build the extrinsic camera matrix from the. We use a pinhole camera model.
  \section{Discussion on the original paper}
\label{sec:discussion}
ADOP\cite{Aruckert2022adop} is overall an excellent paper. It does not bring so much novel ideas but makes a considerable engineering effort to apply the core idea of NPBG \cite{Aliev2020} to large real scenes.
In section ~\ref{subsec:improvement_ideas}, I will discuss some improvement ideas I had while working on the re-implementation of the ADOP paper. But we'll discuss a point of criticism in the next section.\\
\noindent \textbf{~\ref{subsec:limits_real_scenes}{: Real scenes $\approx$ biased evaluation}}. Evaluation on real scenes from the Tanks and Temple dataset \cite{Knapitsch2017TanksAndTemples} mixes two things:
\begin{itemize}
    \item the inherent neural rendering method (point based rendering) quality assesment.
    \item all improvements made by taking into account the camera pipeline.
\end{itemize}
Despite ablation studies to see the effect taking the camera photo pipline into account, the comparison to other methods is unfair and I'll propose a few ideas for a new benchmark to evaluate neural rendering quality in a more controlled environment for research purpose.

\subsection{Limitation: Benchmarking only on real scenes}
\label{subsec:limits_real_scenes}
\noindent \textbf{Focusing only on real scenes.}
Suprisingly, the authors do not mention any attempts to apply their method to synthetic scenes, like the NERF paper initially did and they mention that training a new scene requires a large amount of compute. This could mean they developped their method progressively on a few toy examples and didn't bother releasing results of these tiny examples. Or their method may simply not perform too well on synthetic scenes... which may include a lot of specular materials \footnote{Classical NERF test samples scenes usually contain a lot of specular materials where the rendering would most probably fail}. I still think it would be beneficial to have a few synthetic scenes to test on including to make quick tests without the need of A100 40Gb trained for several hours.
On the other hand, since the main contribution are refining pose estimation (requiring the approximate differentiable aspect of the renderer), handling large scenes and their camera simulator, it's not surprising that the authors showcase their work on natural photos rather than simulation. 

\noindent \textbf{Camera pipeline module.}
The attempt of the authors to model the camera pipeline comes from a good intention. By including the camera pipeline as part of their rendering, they are able to: 

\noindent - work in a linear color space where... performing additions or convolutions is physically grounded.

\noindent - compensate exposure and colors shifts per scene.

\noindent - get a better accuracy on the Tanks and Temple benchmark which has a lot of exposure changes.

\noindent I think the authors picked the low hanging fruits here: take care of most exposure changes to end up improving the metrics on the available benchmark: Table 4. from their paper shows that ADOP is better by a large margin than all other algorithms (except the M60 tanks scene which has been shot in manual frozen exposure). 
They didn't backport their idea to the comcurrent methods they compare to so I don't think the comparison is really fair (the point based method may simply not be the key element leading to good results).

\noindent  Anyway, their claim to handle the camera imaging pipeline properly is legitimate, extremely well crafted and justified in the paper. But it has a major caveat: results may crumble in case of a more complex or different camera ISP \footnote{Image Signal Processor} than the one from Sony A7SII or DJI X5R.
Please refer to ~\cref{fig:ISP} to compare a standard ISP pipeline against the ADOP rendering.

\noindent The idea is interesting but it will most probably fail with a modern smartphone camera which use sophisticated local correction. Other kind of photo finish effects such as vibrancy make some colors slightly more saturated (e.g. blue for skies but avoid saturating red too much for skin tones), tone mapping may act locally \cite{aubry2014fast} and sharpening may be region dependent. The authors show that they're able to get a sky with cyan shift which looks like the camera picture (they fit the tone mapper correctly). Modern smartphone ISPs manufacturers have worked on this "cyan cast" problem (compress blue highlights instead of clipping when applying white balance) so \textbf{modeling these algorithms will become harder and harder as they get more and more sophisticated.}


\begin{figure*}[h]
    \centering
    \includegraphics[width=0.9\textwidth]{figures/isp_pipeline_VS_ADOP.png}
    \caption{Comparing a modern ISP camera pipeline at the top versus the ADOP pipeline at the bottom which includes some learnable parameters such as the tone mapping curve, exposure compensation or white balance adjustments.}
    \label{fig:ISP}
\end{figure*}

\noindent Ablation study from the authors (table 3 and figure 6 of \cite{Aruckert2022adop}) shows that ADOP results get better when including the camera module in the pipeline.

\noindent If we carefully look at the Tanks and Temples dataset \cite{Knapitsch2017TanksAndTemples}, pictures are frames extracted from a mp4 video from 2 different high end cameras, sometimes using auto-exposure (see ~\cref{fig:tank_and_temples}). Although it's a good initiative for researchers to keep on using fixed established benchmarks, Tanks and Temples was initially designed for 3D reconstructions (e.g. evaluate performances of Structure from motion like COLMAP estimation compared to a groundtruth lidar captured point cloud) rather than novel view synthesis. 


\begin{figure}[H]
    \centering
    \includegraphics[width=0.45\textwidth]{figures/tanks_and_temples.png}
    \caption{Information on the scenes captured for the Tanks and Temples dataset with high end video cameras in addition to a Lidar point cloud. On the left, the "train" scene shows clear signs of overexposure. The playground scene on the right has an overall correct and steady exposure.}
    \label{fig:tank_and_temples}
\end{figure}


\noindent My recommendation is that all evaluations better be led in the linear domain. It is the best way to make fair comparisons between novel view synthesis algorithms. Benchmarking would strongly benefit from a dataset "upgrade": a sort of \textit{"Linear Tanks and Temples"} carefully crafted by taking RAW shots with a high end DSLR. This is proposed next.

\noindent \textbf{Switching to RAW format?} 
One of the potential way of creating a new benchmark would be to capture the scenes with a DSLR (like a full frame sensor) shot both in RAW and jpg (usually available on most cameras). RAW files would be post-processed by Adobe LightRoom or DxO Photolab with a neutral rendering: no tone mapping and neutral color rendering to get a Linear RGB set of images without any image compression. In case the sensor 12 or 14bits dynamic range is not sufficient, HDR captures could be achieved by bracketing. Intuitively, it sounds natural to use a tripod and merge the LDR linear images into a HDR linear image. These HDR images may serve as the ground truth for validation.
\noindent But even handheld, all raw captures with bracketing / at various exposures could be used (let the alignment of bracketing images be implicitly done by the 3D rendering engine, without any explicit need to merge exposures).

The main caveat of using RAW images is that there'll always be noise present (proper to the camera sensor) in the raw files. Since you get multiple views of the same scene, we can use the key concept proposed in Noise2Noise \cite{lehtinen2018noise2noise} that a neural network can be trained to denoise images on noisy images directly (without clean ground truth images - it requires noisy burst of the same image instead). Here we have access to the same scene under different angles. The engine to implicity align them and get the right supervision is the rendering of the point cloud itself. The idea of using NERF to work on RAW data has been proposed in "NERFs in the dark" \cite{mildenhall2022rawnerf}.


\noindent RGB linear format is the only space in which an evaluation of the rendering quality does not depend on the camera ISP. Supervision could be linear RGB (denoised and demosaicked RAW files)... or simply RAW files (see the pioneer idea in Mosaic2Mosaic \cite{ehret2019joint}).


\subsection{Improvement ideas}
\label{subsec:improvement_ideas}
A lot of my implementation work has been based on trying to apply the concepts of the ADOP paper to calibrated scenes. I think that it is possible to make these photorealistic images look more like the ones we'd get from real cameras.

\subsubsection*{Calibrated scenes + realistic simulated camera pipeline}.
\noindent Regarding my simplified implementation, I did not try to include the camera pipeline module not only because of limited time but also because simulating a realistic camera pipeline is a very difficult task.
Starting from calibrated scenes, here are the steps to mimick a realistic camera pipeline:
\begin{itemize}
    \item output HDR linear images out of Blender \footnote{\texttt{BlenderProc} does not even offer EXR outputs at the moment}.
    \item mimick RAW files: inverse the "blue linear" blocks of ~\cref*{fig:ISP}. The idea of reversing the ISP has been proposed in \cite{brooks2018unprocessing}: apply inverse white balance and color matrices from typical values, mosaick, add realistic poisson+gaussian noise depending on sensor characteristics and ISO value...
    \item Mimick the ISP which goes from 12/14bits linear RAW bayer data to a 8 bit jpg (which is a difficult task). One could simply use a high end software raw converter by re-introducing a DNG as input to Darktable (open source) or Adobe LightRoom. Another way is to use a deep neural network proxy of a blackbox ISP (see the work \cite{ignatov2020replacing} which mimicks the Huawei P20 ISP)
\end{itemize}
This is definitely \textbf{a large chunk of work} but it could bring a lot of value if one wants to try to make a commercial product for novel view synthesis which tries to adapt to all sorts of cameras.

\subsubsection*{Pseudo colors intialization}.
Instead of initializing the pseudo colors of each point randomly, a trick could be to pre-train an auto-encoder on the reference images. For each point in the point cloud projected onto a given view, we have a target patch in the reference view. This patch could simply be encoded into a latent vector which could be aggreagated in the pseudo-color vector. Pushing the idea further, the pretrained decoder (from the auto-encoder) coud be used as the decoder in the neural rendering network. Here's the intuition: if the neural render encoder is the identity, this is equivalent to copy pasting the decoded patches at the right locations of projected points. Fine tuning the network will finish the job of stitching the patches together.

\subsubsection*{Inherent limitations to model view dependant material appearance}.
By construction, the ADOP pipeline does not have a natural ability to model view dependent effects such as specularities or reflections. We have observed this in ~\cref{fig:view_dependancy_issue}.
We could provide extra inputs to the neural network to model view dependent materials. By providing the angle between the point normal and the view direction (these 2 vectors are already computed during the normal culling test in the first hard depth test pass). These 2 angles could be concatenated to the projected pseudo colors vectors (and we'd probably need to map these to sine positional embeddings as proposed in NERF  \cite{mildenhall2020nerf}). \textit{I eventually found out about NPBG++ \cite{rakhimov2022npbg}}. The authors of NPBG++ propose to fit the coefficients of spherical harmonics function which model color variation of each point with regard to relative camera operation.

\subsubsection*{Limitation when zooming too much}.
The inherent limitation of point appears when you zoom in too close. A continuous parametric representation such as an anisotropic 2D Gaussian Kernel would allow mixing the advantage of point clouds (scale and speed) while trying to remove their discrete nature... Gaussian splatting \cite{kerbl3Dgaussians} seems to be one of the most appropriate answer to this problem  and removes the use of a CNN.
  \section{Conclusion}
\label{sec:conclusion}

In conclusion, the review of this paper has been a great opportunity to dive into the world of neural rendering and understand the challenges of point based neural rendering.

\noindent I have implemented two major items: 
\begin{itemize}
    \item A script to render calibrated scenes which are ready for neural rendering (with perfect normals, perfect point cloud position and camera poses). 
    \item A pytorch implementation of point projections with the fuzzy depth test (which was satisfying and fast enough to perform live inference)
\end{itemize}
I was therefore able to train a simple multiscale decoder CNN jointly with point pseudo colors on simple calibrated scene.

\noindent I have pointed out:
\begin{itemize}
    \item a minor criticism regarding all tricks deployed to model the camera pipeline instead of truly questioning the relevance of benchmarks on the Tanks and Temple dataset. 
    \item a limitation regarding handling view dependent effects which is not a problem in NERF.
\end{itemize}

\noindent The amount of work spent by the authors to make their method work very fast on large scale point clouds is tremendous. The crafstmanship of the computer graphics research community is very inspiring.



\begin{figure}[H]
    \centering
    \includegraphics[width=0.4\textwidth]{figures/inference_live.png}
    \caption{Live inference, the user can move the camera around the scene and see the novel view rendered in real time. On the right side we visualize the projected pseudo colored point cloud of dimension 8 using a Principal Component Analyzis.}
    \label{fig:live_inference}
\end{figure}

  \newpage

  %% APPENDIX
  \appendix
  \section{Appendix}


\begin{figure}[H]
    \centering
    \includegraphics[width=0.45\textwidth]{figures/inference_live.png}
    \caption{Live inference, the user can move the camera around the scene and see the novel view rendered in real time. On the right side we visualize the projected pseudo colored point cloud of dimension 8 using a Principal Component Analyzis.}
    \label{fig:live_inference}
\end{figure}



\begin{table*}[htpb]
    \begin{tabular}{|l|l|l|l|l|l|}
    \hline
    Number of points & PSNR   & Mode     & Convolutions size & Pseudo color dimension & Multi-scale supervision \\ \hline
    100k             & 16.7dB & Bypass   & 1x1               & 3                      & Yes                     \\ \hline
    100k             & 21.2dB & Conv 5x5 & 5x5               & 3                      & Yes                     \\ \hline
    100k             & 26.3dB & Vanilla  & 5x5               & 8                      & No                      \\ \hline
    100k             & \textbf{29dB}   & Vanilla  & 5x5               & 8                      & Yes                     \\ \hline
    400k             & 23dB   & Bypass   & 1x1               & 3                      & No                      \\ \hline
    800k             & 25db   & Bypass   & 1x1               & 3                      & No                      \\ \hline
\end{tabular}
\caption{We report quantitative performances (PSNR on validation set) of various training configurations on the \texttt{Old chair} scene.}
\label{tab:results}
\end{table*}
  
  \newpage
  %% BIBLIOGRAPHY
  \bibliographystyle{ACM-Reference-Format}
  \bibliography{references}

\end{document}
